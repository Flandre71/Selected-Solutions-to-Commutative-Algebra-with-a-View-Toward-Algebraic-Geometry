\documentclass{article}
\usepackage{geometry}
\geometry{a4paper,scale=0.8}
\usepackage[utf8]{inputenc}

\usepackage{amsmath, amssymb}
\usepackage{amsthm}
\usepackage{enumitem}
\newtheoremstyle{breakstyle}
	{5pt}
	{10pt}
	{\normalfont}
	{0pt}
	{\bfseries}
	{}
	{\newline}
	{}
\theoremstyle{breakstyle}
\newtheorem{exercise}{Exercise}[section]
\title{Chapter 1 - Roots of Commutative Algebra}
\setlist[enumerate,1]{
    label=(\roman*), 
    topsep=0pt, 
    partopsep=0pt, 
    parsep=0pt,
    leftmargin=*
}

\author{Flandre}
\date{15 Jan 2026}

\begin{document}
\maketitle

\setcounter{section}{1}
\begin{exercise}
	\leavevmode \vspace{-\baselineskip}
	\begin{enumerate}
		\item \emph{(1)}$\Rightarrow$\emph{(2)}:
			Suppose there exist a chain $\{N_{k}\}_{k=1}^{\infty}$ which is a none stopping ascending chain, we denote 
			$$
			\bigcup_{k =1} ^{\infty}N_{k}=N
			$$ 
			and by \emph{(1)}, $N$ is a finitely generately submodule of $M$.
			Suppose $\left\{ n_1,n_2,\ldots,n_{r} \right\} $ is the generator of $N$, and each $n_{k}$ is in one of the submodule ${N_{i}}$.
			Now by the infinte ascending chain, there exist $N_{j}$ such that $N_{j}\supseteq {n_1,\ldots,n_{r}}$.
			And we pick out
			$$
			n \in N_{j+1}\backslash N_{j}
			$$
			and notice $n \in N$, contradictory!
		\item \emph{(2)}$\Rightarrow$\emph{(3)}:
			Denote such set of submodules by $\Sigma$.

			First we pick out $N_1\in \Sigma$. If $N_1$ is maximal, \emph{(2)} holds.
			Otherwise $\exists N_2\in \Sigma,N_2\supseteq N_1$. If $N_2$ is maximal, again  \emph{(3)} holds, otherwise $\exists N_3\in \Sigma,N_3\supseteq N_2$.
			Under such method, we can derive a chain $N_1,N_2,N_3,\ldots$ which satisfy the A.C.C., therefore by \emph{(2)} it must terminate.
		\item \emph{(3)}$\Rightarrow$\emph{(1)}:
			For any submodule $N$, let $\Sigma$ be the set of all finitely generated submodules of $N$.
			Since  $\left\{ 0 \right\} \subseteq\Sigma$, $\Sigma$ is nonempty.

			Now we pick out $N_0$ to be the maximal element of $\Sigma$.
			If $N=N_0$, \emph{(1)} holds.
			Otherwise let $n\in N\backslash N_0$.
			Notice the base of $N_0$, together with $\left\{ n \right\} $, generate a finitely generated submodule of $N$, which contradict with the maximality of $N_0$.
		\item \emph{(2)}$\Rightarrow$ \emph{(4)}:
			Let $\left\{ N_{k} \right\} _{k=1}^{\infty}$ be a sequence of submodules, whose $k $-th term is generated by $\left\{ f_1,f_2,\ldots,f_{k} \right\} $.
			With out loss of generality we assume $N_{k}\subsetneq N_{k+1},\forall k$.
			By \emph{(2)}, the chain $\left\{ N_{k} \right\} $ must termiante, therefore the condition in \emph{(4)} holds.
		\item \emph{(4)}$\Rightarrow$\emph{(2)}:
			Suppose there is a chain of submodules $\left\{ N_{k} \right\} $, we'll prove it'll termiante somewhere.

			For each $k\in \mathbb{N}$, pick out $f_{k}\in N_{k}\backslash N_{k-1}$.
			Now by \emph{(4)},
			$$
			\exists m\in \mathbb{Z}^{+},s.t.\ \forall n>m,\ \exists a\in \mathbb{R},s.t.\ f_{n}=\sum_{k=1}^{m} a_{k}f_{k},\ \mathrm{for}\ a_{k}\in R
			$$ 
			Therefore the ascending chain $\left\{ N_{k} \right\} $ termiante!.
	\end{enumerate}
	\qed
\end{exercise}

\end{document}
