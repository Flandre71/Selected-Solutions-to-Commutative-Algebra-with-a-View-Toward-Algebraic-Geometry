\documentclass{article}
\usepackage{geometry}
\geometry{a4paper,scale=0.8}
\usepackage[utf8]{inputenc}

\usepackage{amsmath, amssymb}
\usepackage{amsthm}
\usepackage{enumitem}
\newtheoremstyle{breakstyle}
	{5pt}
	{10pt}
	{\normalfont}
	{0pt}
	{\bfseries}
	{}
	{\newline}
	{}
\theoremstyle{breakstyle}
\newtheorem{exercise}{Exercise}[section]
\title{Chapter 1 - Roots of Commutative Algebra}
\setlist[enumerate,1]{
    label=(\roman*), 
    topsep=0pt, 
    partopsep=0pt, 
    parsep=0pt,
    leftmargin=*
}

\author{Flandre}
\date{15 Jan 2026}

\begin{document}
\maketitle

\setcounter{section}{1}
\begin{exercise}
	\leavevmode \vspace{-\baselineskip}
	\begin{enumerate}
		\item \emph{(1)}$\Rightarrow$\emph{(2)}:
			Suppose there exist a chain $\{N_{k}\}_{k=1}^{\infty}$ which is a none stopping ascending chain, we denote 
			$$
			\bigcup_{k =1} ^{\infty}N_{k}=N
			$$ 
			and by \emph{(1)}, $N$ is a finitely generately submodule of $M$.
			Suppose $\left\{ n_1,n_2,\ldots,n_{r} \right\} $ is the generator of $N$, and each $n_{k}$ is in one of the submodule ${N_{i}}$.
			Now by the infinte ascending chain, there exist $N_{j}$ such that $N_{j}\supseteq {n_1,\ldots,n_{r}}$.
			And we pick out
			$$
			n \in N_{j+1}\backslash N_{j}
			$$
			and notice $n \in N$, contradictory!
		\item \emph{(2)}$\Rightarrow$\emph{(3)}:
			Denote such set of submodules by $\Sigma$.

			First we pick out $N_1\in \Sigma$. If $N_1$ is maximal, \emph{(2)} holds.
			Otherwise $\exists N_2\in \Sigma,N_2\supseteq N_1$. If $N_2$ is maximal, again  \emph{(3)} holds, otherwise $\exists N_3\in \Sigma,N_3\supseteq N_2$.
			Under such method, we can derive a chain $N_1,N_2,N_3,\ldots$ which satisfy the A.C.C., therefore by \emph{(2)} it must terminate.
		\item \emph{(3)}$\Rightarrow$\emph{(1)}:
			For any submodule $N$, let $\Sigma$ be the set of all finitely generated submodules of $N$.
			Since  $\left\{ 0 \right\} \subseteq\Sigma$, $\Sigma$ is nonempty.

			Now we pick out $N_0$ to be the maximal element of $\Sigma$.
			If $N=N_0$, \emph{(1)} holds.
			Otherwise let $n\in N\backslash N_0$.
			Notice the basis of $N_0$, together with $\left\{ n \right\} $, generate a finitely generated submodule of $N$, which contradict with the maximality of $N_0$.
		\item \emph{(2)}$\Rightarrow$ \emph{(4)}:
			Let $\left\{ N_{k} \right\} _{k=1}^{\infty}$ be a sequence of submodules, whose $k $-th term is generated by $\left\{ f_1,f_2,\ldots,f_{k} \right\} $.
			With out loss of generality we assume $N_{k}\subsetneq N_{k+1},\forall k$.
			By \emph{(2)}, the chain $\left\{ N_{k} \right\} $ must termiante, therefore the condition in \emph{(4)} holds.
		\item \emph{(4)}$\Rightarrow$\emph{(2)}:
			Suppose there is a chain of submodules $\left\{ N_{k} \right\} $, we'll prove it'll termiante somewhere.

			For each $k\in \mathbb{N}$, pick out $f_{k}\in N_{k}\backslash N_{k-1}$.
			Now by \emph{(4)},
			$$
			\exists m\in \mathbb{Z}^{+},s.t.\ \forall n>m,\ \exists a\in \mathbb{R},s.t.\ f_{n}=\sum_{k=1}^{m} a_{k}f_{k},\ \mathrm{for}\ a_{k}\in R
			$$ 
			Therefore the ascending chain $\left\{ N_{k} \right\} $ termiante!
	\end{enumerate}
	\qed
\end{exercise}

\begin{exercise}
	Suppose by contradictory that the proposition is wrong, and we denote the set of ideal having infinte amount of minimal prime ideals by $\Sigma$.

	Since $N$ is Noetherian, we may pick out a maximal ideal $I\in \Sigma$.
	If $I$ is prime, $I$ is the unique minimal prime ideal over itself, contradictory.
	On the otherhand, if $I$ is not prime. Then
	$$
	\exists a,b\in R,s.t.\ ab\in I,a,b \not\in I
	$$ 
	Consider $J_1=(I,a)$ and $J_2=(I,b)$.
	For any minimal prime ideal $J\supseteq I$, since it's prime, $ab\in I\subseteq J$ must induce $a\in J$ or $b\in J$, i.e. $J$ is a minimal prime over $J_1$ or $J_2$.

	Because there're infinte many amount of $J$, we see there'll be at least one of the ideal among $J_1,J_2$, which has infinte amount of minimal prime ideals, contradicting the maximality of $I$.
	\qed
\end{exercise}

\begin{exercise}
	\leavevmode \vspace{-\baselineskip}
	\begin{enumerate}
		\item $(\Rightarrow)$:
			When $M$ is Noetherian, $M'$ must be Noetherian.
			By the \emph{Lattice Isomorphism Theorem}, there is a bijection
			$$
			\mathrm{submodules\ of\ }M\mathrm{\ that\ contains\ M'}\longleftrightarrow\mathrm{submodules\ of\ }M / M'
			$$ 
			Now for any submodules of $M / M'$ let $N$ be the correspond submodule of $M$ which contians $M'$.
			Consider
			$$
			(m_1,\ldots,m_{n})
			$$
			to be the generators of $n$.
			Now for $\overline{N}=N / M'$, its generator will be
			$$
			(\overline{m_1},\ldots,\overline{m_{n}})
			$$
			Which is finite.
		\item $(\Leftarrow)$:
			% When $M'$ and $M / M'$ are both Noetherian, consider the map
			% $$
			% \varphi:M\longrightarrow M / M'
			% $$
			% $$
			% m\longmapsto \overline{m}
			% $$
			% Its kernel is $M'$ and its image is $M / M'$.
			% Suppose $M'$ is generated by $\left\{ k_1,\ldots,k_{s} \right\} $ while $M / M'$ is generated by $\left\{  \right\} $
			When $M'$ and $M / M'$ are both Noetherian, for any submodule $N\subseteq M$, we'll prove $N$ is Noetherian.
			\begin{enumerate}
				\item When $N\subseteq M'$, $N$ is Noetherian and it's done.
				\item When $M'\subseteq N$, by \emph{Lattice Isomorphism Theorem} $N$ correspond to $N / M'$, which is a submodule of $M / M'$.
					Suppose $N / M'$ is generated by
					$$
					(\overline{m_1},\ldots,\overline{m_{n}})
					$$
					Of each generators above, we consider its preimage, which is the coset $m_1+M'$.
					Since $M'$ is finitely generated, those cosets can be finitely expressed, and $N$ is finitely generated.
				\item When $M'\backslash N,N\backslash M$ are both nonempty, consider $N\backslash M'$ quotient by $N\cap M'$, similar the the case \emph{(b)}.
			\end{enumerate}
	\end{enumerate}
	\qed
\end{exercise}

\begin{exercise}
	\leavevmode \vspace{-\baselineskip}
	\begin{enumerate}[label=({\roman*})]
		\item $(1)\Rightarrow(2):$
			For any ideal $I_0\subseteq R_0$, we need to show $I_0$ is finitely generated.

			Consider $I_0R$, it is an ideal in $R$, thus finitely generated.
			Suppose $I_0R$ is generated by
			$$
			G=\left\{ g_1,g_2,\ldots,g_{n} \right\} 
			$$
			For $\forall k\in [1,n]\cap \mathbb{Z},g_{k}\in G$, suppose after direct sum decomposition, it was decompsed into $g_{k}^{(0)}+g_{k}^{(1)}+\ldots$.
			We'll prove $I_0$ was generated by $\left\{ g_{1}^{(0)},g_{2}^{(0)},\ldots,g_{n}^{(0)} \right\} $.
			Indeed, for $\forall r_0\in I_0$, as an element in $I$ it can be expressed by
			$$
			r_0=s_1g_1+s_2g_2+\ldots+s_{n}g_{n}
			$$
			where $\forall k\in [1,n]\cap \mathbb{Z},s_{k}\in R$.
			We follow the notation above and decompose $s_{k}$ into
			$$
			s_{k}=s_{k}^{(0)}+s_{k}^{(1)}+\ldots
			$$
			Now
			\begin{align*}
				r_0
				&= \sum_{k=1}^{n} s_{k}g_{k}  \\
				&= \sum_{k=1}^{n} \left( \sum_{i=0}^{\infty} s_{k}^{(i)} \sum_{j=0}^{\infty} g_{k}^{(j)}\right)    \\
				&= \sum_{k=1}^{n} s_{k}^{(0)}g_{k}^{(0)}+\text{(rest of the terms)}\\
			\end{align*}
			Since $\forall i,j\in \mathbb{Z}^{+}\cup\left\{ 0 \right\} ,R_{i}\cap R_{j}={0}$, we know each of the terms rest in the above does not lies in $R_0$.
			So they sum up to $0$, i.e.
			$$
			r_0=\sum_{k=1}^{n} s_{k}^{(0)}g_{k}^{(0)} 
			$$
			So $R_0$ is generated by $\left\{ g_1^{(0)},g_2^{(0)},\ldots,g_{n}^{(0)} \right\} $.
			Lastly, $R_1\oplus R_2\oplus\ldots$ which obviously is a ideal, is finitely generated by the Noetherian property of $R$.

			\emph{(A smarter approach: consider the homomorphism mapping $R_0$ to $R_0$ and the rest to $0$, it will map $R$ to $R_0$ and map $G$ to the bases we require.)}
		\item $(2)\Rightarrow(3):$
			We need to show $R$ is a finitely generated $R_0$-algebra:

			Notice that addition in $\bigoplus_{k=1}^{\infty}R_{k}$ is naturally derived from the additive structure of rings $R_1,R_2,\ldots$ and direct sum.
			Also, for scaler multiplication, since
			$$
			R_0R_{k}\subseteq R_{k}
			$$
			We have
			\begin{align*}
				r_0\left( \bigoplus_{k=1}^{\infty}R_{k} \right) 
				&= \bigoplus_{k=1}^{\infty}r_0R_{k} \\
				&\subseteq \bigoplus_{k=1}^{\infty}R_{k} \\
			\end{align*}
			where $r_0\in R_0$.
			From which we naturally obtain a scaler multiplication structure
			$$
			R_0\times \bigoplus_{k=0}^{\infty}R_{k}\longrightarrow \bigoplus_{k=0}^{\infty}R_{k}
			$$
			Because $\bigoplus_{k=1}^{\infty}R_{k}$ is finitely generated, $R=\bigoplus_{k=0}^{\infty}R_{k}$ is a finitely generated $R_0$-algebra.
			% Since $R_0$ is Noetherian and $R_1\oplus R_2\oplus\ldots$ is finitely generated,
			% $$
			% R=R_0\oplus R_1\oplus R_2\oplus\ldots
			% $$
			% can be viewed as a finitely generated $R_0$-algebra.
			% Therefore by \emph{Corollary 1.3}, $R$ is Noetherian.
		\item $(3)\Rightarrow(1):$
			This is just the second statement of \emph{Corollary 1.3}.
	\end{enumerate}
	\qed
\end{exercise}

\begin{exercise}
	We use the method of the solution at \emph{Page.712}.

	For any ascending chain of ideals $I_{0}\subseteq I_{1}\subseteq I_{2}\subseteq \ldots$ of $R$,
	consider ascending chain of ideals
	$$
	I_0S\subseteq I_1S\subseteq I_2S\subseteq \ldots
	$$
	of S.
	Since $S$ is Noetherian, $\exists N\in \mathbb{Z}^{+},\mathrm{s.t.}\ \forall i,j>N,I_{i}S=I_{j}S$.
	Therefore $\pi(I_{i}S)=\pi(I_{j}S)$, i.e.
	$$
	I_{i}\pi(S)=I_{j}\pi(S)
	$$
	$$
	I_{i}=I_{j}
	$$
	\qed
\end{exercise}

\begin{exercise}
	\leavevmode \vspace{-\baselineskip}
	\begin{enumerate}[label=({\alph*})]
		\item Because for $S_{A}=\left\{ B\in S\mid \mathrm{deg}(A)\le \mathrm{deg}B \right\} ,\left| S_{A} \right| <\infty$.
			Also, $\forall B\in S_{A},B\neq A$ we must have $B<A$ or $B>A$.
			So the number of monomials $B$ such that $A>B$ must be finite.
		\item When $p$ is invariant under $\Sigma$.
			For $\forall \sigma\in \Sigma$ and any monomial term $s$ of $p$, $\sigma(s)$ must be a term in $p$.
			
			Hence, if $x_1^{m_1}x_2^{m_2}\ldots x_{r}^{m_{r}}$ is a term in $p$, $x_1^{\sigma^{-1}(m_1)}x_2^{\sigma^{-1}(m_2)}\ldots x_{r}^{\sigma^{-1}(m_{r})}$is a term in $p$.
			In the polynomial
			$$
			\sum_{\sigma\in \Sigma}^{} \sigma(x_1^{m_1}\ldots x_{r}^{m_{r}}) 
			$$
			we can easily see that the leading term will be
			$$
			x_1^{m_1'}\ldots x_{r}^{m_{r}'}
			$$
			Where $\left( m_1',\ldots ,m_{r}' \right) $ is a permutation of $\left( m_1,\ldots ,m_{r} \right) $ and $m_1'\ge m_2'\ge \ldots \ge m_{r}'$.
			Notice
			 $$
			p=\frac{1}{\left| \Sigma \right| }\sum_{\sigma\in \Sigma}^{} \sigma(p) 
			$$
			Which is to say any monomials of $p$ under $\Sigma$, the leading term of its sum of orbits must have the degree in the above format.
			Therefore we can see the leading term of $p$, being selected from sum of the orbits under $\Sigma$ of monomials $m$, must be in the form $x_1^{m_1}\ldots x_{r}^{m_{r}}$ where $m_1\ge \ldots \ge m_{r}$.
		\item Just compute directly, trivial.
		\item Since $m_{i}=\sum_{j\ge i}^{r} \mu_{j} $, $\mu_{i}=m_{i}- m_{i+1}$ (we define $m_{r+1}=0$).
			Hence, $(\mu_1,\ldots ,\mu_{r})$ is uniquely defined by $(m_1,\ldots ,m_{r})$, so it's an injection (monomorphism).

			Also, indeed the monomial $x_1^{m_1}\ldots x_{r}^{m_{r}}$ with $m_1\ge \ldots \ge x_{r}$ is a initial term of $f_1^{\mu_1}\ldots f_{r}^{\mu_{r}}$.
		\item For any $g\in S^{\Sigma}$, we can write it uniquely in the form
			$$
			g=\lambda_1\sum_{\sigma\in \Sigma}^{} \sigma(g_1)+\lambda_2\sum_{\sigma\in \Sigma}^{} \sigma(g_2)+\ldots +\lambda_{m}\sum_{\sigma\in \Sigma}^{} \sigma(g_m)
			$$
			Where each of those $g_{k}=x_{1}^{m_{k,1}}\ldots x_{r}^{m_{k,r}}$ satisfies $m_{k,1}\ge \ldots \ge m_{k,r}$, and for $\forall i\neq j, g_{i}\neq g_{j}$.
			Moreover, from \emph{(a)$\sim$(d)} we know
			\begin{align}
				\sum_{\sigma\in \Sigma}^{} g_{k} 
				&= \sum_{\sigma\in \Sigma}^{}  x_{k}^{m_{k,1}}\ldots x_{k,r}^{m_{k,r}} \\
				&= f_1^{\mu_{k,1}}\ldots f_{r}^{\mu_{k,r}}
			\end{align}
			since both \emph{(1)} and \emph{(2)} has the same leading term, both being symmetric and share the same coefficient.
	\end{enumerate}
	\qed
\end{exercise}

\begin{exercise}
	\leavevmode \vspace{-\baselineskip}
	\begin{enumerate}[label=({\alph*})]
		\item 
			\begin{enumerate}[label=({\roman*})]
				\item For any polynomial $p\in k[x,y],\mathrm{s.t.}\ $, we have
					$$
					\mathrm{deg}_{x}p+\mathrm{deg}_{y}p\equiv 0(\mathrm{mod}2)
					$$
					Therefore we can represent $p$ by $x^2,y^2,xy$.
					Since $p$ is chosen arbitarily,
					$$
					\text{ring of invariants}\subseteq k[x^2,xy,y^2]
					$$
					Also, from $x^2,xy,y^2$ being invariant under $g$, we know 
					$$
					k[x^2,xy,y^2]\subseteq \text{ring of invariants}
					$$
					So the ring of invariants is $k[x^2,xy,y^2]$.
				\item Consider ring homomorphism
					$$
					\varphi:k[u,v,w]\longrightarrow k[x^2,xy,y^2]
					$$
					where
					\begin{align*}
						u\longmapsto x^2\\
						v\longmapsto xy\\
						w\longmapsto y^2
					\end{align*}
					Notice $\ker \varphi=(uw-v^2)$, so the proof can be obtained from \emph{First Ring Isomorphism Theorem}.
				\item $g$ acts by sending $x,y$ to $-x,-y$ respectively.
					Thus, for any plane $p$ on the corresponding affine $2$-space $\mathbb{A}^2$, it must be affected by the operation $g$, ending up not being identity.
			\end{enumerate}
		\item 
			$G$ acts by $g(x_{i})=\alpha_{i}(g)x_{i}$ for $\forall g\in G,\forall i\in [1,r]\cap \mathbb{Z}$.
			% The invariants of $G$ is the invariants for all those $g\in G$, so we may check the invariants for a specific $g\in G$ first.

			For any monomial $p\in k[x_1,\ldots ,x_{r}]^{G}$, let
			$$
			p=x_1^{a_1}\ldots x_{r}^{a{r}}
			$$
			Since for $\forall g\in G$, $g(p)=p$, we have
			$$
			\left( \alpha_1(g)^{a_1}\ldots \alpha_{r}(g)^{a_{r}} \right) 
x_1^{a_1}\ldots x_{r}^{a_{r}}
			=x_1^{a_1}\ldots x_{r}^{a_{r}}
			$$
			and we get
			$$
			\alpha_1(g)^{a_1}\ldots \alpha_{r}(g)^{a_{r}} =1
			$$
			In this way we obtain a homomorphism
			\begin{align*}
				\varphi_{g}:\mathbb{Z}^{r}&\longrightarrow k^{\times }\\
				(a_1,\ldots ,a_{r})&\longmapsto \alpha_1(g)^{a_1}\ldots \alpha_{r}(g)^{a_{r}}
			\end{align*}
			whose kernel is the invariants under $g$.
			Ultimately, the desired invariants undre $G$ can be derived from
			$$
			\bigcap_{g \in G} \ker \varphi_{g}
			$$
	\emph{(I am wondering where was the condition $\left| G \right|<\infty $ used.)}
	\end{enumerate}
	\qed
\end{exercise}

\end{document}
